\documentclass{article}
\usepackage{geometry}
\usepackage{amsmath, amssymb, amsthm}

\newcommand{\Ac}{\mathcal{A}}  % alphabet
\newcommand{\Lc}{\mathcal{L}}  % label function
\newcommand{\Gc}{\mathcal{G}}  % labeled graph
\newcommand{\Hc}{\mathcal{H}}  % labeled graph
\newcommand{\Vc}{\mathcal{V}}
\newcommand{\Ec}{\mathcal{E}}
\newcommand{\shift}[1]{\mathsf{X}_{#1}}
\newcommand{\term}[1]{\textit{#1}}

\newtheorem{theorem}{Theorem}
\newtheorem{definition}[theorem]{Definition}

\begin{document}
    \begin{theorem}
        If \(\mathcal{G} = (G, \mathcal{L})\) is the minimizing right resolving
        presentation of an irreducible sofic shift \(X\) and \(X\) is and
        \(N\)-step shift of finite type, then \(\shift{G} \cong \shift{\Gc}\).
    \end{theorem}
    
    \begin{proof}
        Let \(x, y\) be walks in \(\shift{G}\). If \(\mathcal{L}_\infty(x)=\mathcal{L}_\infty(y)\),
        then for any \(i\), the paths \(x_{[i-n, i-1]}\) and \(y_{[i-n, i-1]}\) present 
        the same word. Because that word is of length \(N\), the word is synchronizing
        for \(\mathcal{G}\) (from 3.4.17), so those paths end at the same vertex. Since
        \(\mathcal{L}(x_{[i]}) = \mathcal{L}(y_{[i]})\), \(\mathcal{G}\) is right 
        resolving, and \(x_{[i-n, i-1]}\) and \(y_{[i-n, i-1]}\) end at the same vertex,
        then \(x_{[i]} = y_{[i]}\) and hence \(x = y\), so \(\mathcal{L}_\infty\) is injective.
        By definition, \(\mathcal{L}_\infty\) is surjective. Therefore, \(\mathcal{L}_\infty\) is bijective and 
        a conjugacy from \(\mathsf{X}_G\) to \(\mathsf{X}_\mathcal{G}\).
    \end{proof}

    \begin{definition}
        A \term{graph} \(G\) is a finite set of \term{verticies} \(\Vc=\Vc(G)\) and a finite set 
        of edges \(\Ec = \Ec(G)\) with each \(e \in \Ec\) starting at a vertex \(i(e) \in \Vc\)
        and terminating at a vertex \(t(e) \in \Vc\). Note that two edges can start at terminate
        at the same vertex.
    \end{definition}

    \begin{definition}
        Let \(G\) and \(H\) be graphs. A \term{graph isomorphism from \(G\) to \(H\)}
        is a bijective pair of maps \(\partial \Phi : \Vc(G) \to \Vc(H)\) and \(\Phi : \Ec(G) \to \Ec(H)\)
        such that \(i(\Phi(e)) = \partial \Phi(i(e))\) and \(t(\Phi(e)) = \partial \Phi(t(e))\)
        for all \(e \in \Ec(G)\). If there exists a graph isomorphism between \(G\) and \(H\),
        then \(G\) and \(H\) are graph isomorphic and is denoted \(G \cong H\).
    \end{definition}

    \begin{definition}
        Let \(\Gc = (G, \Lc_G)\) and \(\Hc = (H, \Lc_H)\) be labeled graphs.
        A label-graph isomorphism is a graph isomorphism \((\partial \Phi, \Phi) : G \to H\)
        such that \(\Lc_H(\Phi(e)) = \Lc_G(e)\) for all \(e \in \Ec(G)\), which is 
        denoted \((\partial \Phi, \Phi) : \Gc \to \Hc\). If there exists a label-graph 
        isomorphism between \(\Gc\) and \(\Hc\), then \(\Gc\) and \(\Hc\) are label-graph
        isomorphic (or just isomorphic) and is denoted \(\Gc \cong \Hc\).
    \end{definition}

    \begin{theorem} 
        If \((\partial\Phi, \Phi) : G \to H\) is a graph isomorphism from \(G\) to \(H\),
        then \((\partial\Phi^{-1}, \Phi^{-1}): H \to G\) is a graph isomorphism from \(H\) to \(G\).
    \end{theorem}

    \begin{proof}
        For an edge \(e_G \in \Ec(G)\), we have 
        \begin{align*}
            \partial\Phi(i(e)) &= i(\Phi(e))  \\
            \partial\Phi^{-1}(\partial\Phi(i(e))) &= \partial\Phi^{-1}(i(\Phi(e)))\\
            i(e) &= \partial\Phi^{-1}(i(\Phi(e)))
        \end{align*}
        Hence, for an edge \(e_H \in \Ec(H)\), \(\Phi^{-1}(e_H) \in \Ec(G)\) so 
        \begin{align*}
            i(\Phi^{-1}(e_H)) &= \partial\Phi^{-1}(i(\Phi(\Phi^{-1}(e_H))))\\
            &= \partial\Phi^{-1}(i(e_H))
        \end{align*}
        A similar argument shows that \(t(\Phi^{-1}(e_H)) = \partial\Phi^{-1}(t(e))\).
        
    \end{proof}


    \begin{theorem}
        Let \(\Gc = (G, \Lc_G)\) and \(\Hc = (H, \Lc_H)\) be labeled graphs. 
        If \(\Gc \cong \Hc\), then \(\shift{\Gc} = \shift{\Hc}\).
    \end{theorem}

    \begin{proof}
        For \(x \in \shift{\Gc}\), there exists a \(y \in \shift{G}\) such that \(x_i = \Lc_G(y_i) = \Lc_H(\Phi(y_i))\).
        Note that for all \(i \in \mathbb{Z}\),
        \begin{align*}
            t(y_i) &= i(y_{i+1})\\
            \partial \Phi(t(y_i)) &= \partial\Phi(i(y_{i+1}))\\
            t(\Phi(y_i)) &= i(\Phi(y_{i+1}))
        \end{align*}
        so \(\Phi_\infty(y) \in \shift{H}\) and therefore \(x=\left( \Lc_H \circ \Phi \right)_\infty(y) \in \shift{\Hc}\).
    \end{proof}

    
\end{document}