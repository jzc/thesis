\documentclass{article}
\usepackage{geometry}
\usepackage{amsmath, amssymb, amsthm}

\newcommand{\A}{\mathcal{A}}  % alphabet
% \newcommand{\L}{\mathcal{L}}  % label function
\newcommand{\G}{\mathcal{G}}  % labeled graph
\newcommand{\shift}[1]{\mathsf{X}_{#1}}

\newtheorem{theorem}{Theorem}

\begin{document}
    \begin{theorem}
        If \(\mathcal{G} = (G, \mathcal{L})\) is the minimizing right resolving
        presentation of an irreducible sofic shift \(X\) and \(X\) is and
        \(N\)-step shift of finite type, then \(\shift{G} \cong \shift{\G}\).
    \end{theorem}
    \begin{proof}
        Let \(x, y\) be walks in \(\shift{G}\). If \(\mathcal{L}_\infty(x)=\mathcal{L}_\infty(y)\),
        then for any \(i\), the paths \(x_{[i-n, i-1]}\) and \(y_{[i-n, i-1]}\) present 
        the same word. Because that word is of length \(N\), the word is synchronizing
        for \(\mathcal{G}\) (from 3.4.17), so those paths end at the same vertex. Since
        \(\mathcal{L}(x_{[i]}) = \mathcal{L}(y_{[i]})\), \(\mathcal{G}\) is right 
        resolving, and \(x_{[i-n, i-1]}\) and \(y_{[i-n, i-1]}\) end at the same vertex,
        then \(x_{[i]} = y_{[i]}\) and hence \(x = y\), so \(\mathcal{L}_\infty\) is injective.
        By definition, \(\mathcal{L}_\infty\) is surjective. Therefore, \(\mathcal{L}_\infty\) is bijective and 
        a conjugacy from \(\mathsf{X}_G\) to \(\mathsf{X}_\mathcal{G}\).
    \end{proof}
\end{document}